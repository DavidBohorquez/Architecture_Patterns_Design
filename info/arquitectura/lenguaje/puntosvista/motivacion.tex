\subsection{Punto de Vista Motivacional}

\subsubsection{Punto de Vista Implicado}

El punto de vista de la estrategia permite al arquitecto empresarial modelar una descripción general estratégica de alto nivel de las estrategias (cursos de acción) de la empresa, las capacidades y los recursos que las respaldan y los resultados previstos.
\begin{table}[ht]
	
	\begin{center}
		\begin{tabular}{ | m{6em} | m{8cm}|  } 
			
			\hline
			Implicado & Partes interesadas, gerentes de negocios, arquitectos empresariales y TIC, analistas de negocios, gerentes de requisitos 
			\\
			\hline
			Preocupaciones & Misión y estrategia de arquitectura, motivación
			\\
			\hline
			Propósito & Diseñar, decidir, informar
			\\
			\hline
			Alcance & Motivación
			\\
			\hline
		\end{tabular}
		\caption{Punto de Vista Implicado}
		\label{tab:concepts}
	\end{center}
\end{table}

\subsubsection{Elementos}
\begin{itemize}
	\item Implicado
	\item Manejador
	\item Valoración
	\item Objetivo
	\item Resultado
\end{itemize}

\subsubsection{Punto de vista de la realización de objetivos}


El punto de vista del mapa de capacidades permite al arquitecto comercial crear una descripción general estructurada de las capacidades de la empresa. Normalmente, un mapa de capacidades muestra dos o tres niveles de capacidades en toda la empresa. Puede, por ejemplo, usarse como mapa de calor para identificar áreas de inversión. En algunos casos, un mapa de capacidades también puede mostrar resultados específicos entregados por estas capacidades.
\begin{table}[ht]
	\begin{center}
		
		\begin{tabular}{ | m{6em} | m{8cm}|  } 
			\hline
			Implicado & Partes interesadas, gerentes de negocios, arquitectos empresariales y TIC, analistas de negocios, gerentes de requisitos 
			\\
			\hline
			Preocupaciones & Misión de arquitectura, estrategia y táctica, motivación
			\\
			\hline
			Propósito & Diseñar, decidir
			\\
			\hline
			Alcance & Motivación
			\\
			\hline
		\end{tabular}
		\caption{Punto de vista de la realización de objetivos}
		\label{tab:concepts}
	\end{center}
\end{table}

\subsubsection{Elementos}
\begin{itemize}
	\item Objetivo
	\item Principio
	\item Requerimiento
	\item Restricción
	\item Resultado
\end{itemize}

\subsubsection{Punto de vista de la realización de requisitos}


El punto de vista de la realización de resultados se utiliza para mostrar cómo las capacidades y los elementos básicos subyacentes producen los resultados orientados al negocio de más alto nivel.

\begin{table}[ht]
	\begin{center}
		
		\begin{tabular}{ | m{6em} | m{8cm}|  } 
			\hline
			Implicado & Partes interesadas, gerentes de negocios, arquitectos empresariales y TIC, analistas de negocios, gerentes de requisitos 
			\\
			\hline
			Preocupaciones & Misión de arquitectura, estrategia y táctica, motivación
			\\
			\hline
			Propósito & Diseñar, decidir
			\\
			\hline
			Alcance & Motivación
			\\
			\hline
			
		\end{tabular}
		\caption{Punto de vista de la realización de requisitos}
		\label{tab:concepts}
	\end{center}
\end{table}

\subsubsection{Elementos}
\begin{itemize}
	\item Objetivo
	\item Requerimiento / Restricción
	\item Resultado
	\item Valor
	\item Significado
	\item Elemento central
\end{itemize}

\subsubsection{Punto de vista de la motivación} 


El punto de vista del mapa de recursos permite al arquitecto comercial crear una descripción general estructurada de los recursos de la empresa. Un mapa de recursos generalmente muestra dos o tres niveles de recursos en toda la empresa. Puede, por ejemplo, usarse como mapa de calor para identificar áreas de inversión. En algunos casos, un mapa de recursos también puede mostrar relaciones entre los recursos y las capacidades a las que están asignados.

\begin{table}[ht]
	\begin{center}
		
		\begin{tabular}{ | m{6em} | m{8cm}|  } 
			\hline
			Implicado & Partes interesadas, gerentes de negocios, arquitectos empresariales y TIC, analistas de negocios, gerentes de requisitos 
			\\
			\hline
			Preocupaciones & Misión de arquitectura, estrategia y táctica, motivación
			\\
			\hline
			Propósito & Diseñar, decidir
			\\
			\hline
			Alcance & Motivación
			\\
			\hline
		\end{tabular}
		\caption{Punto de vista de la motivación}
		\label{tab:concepts}
	\end{center}
\end{table}

\subsubsection{Elementos}
\begin{itemize}
	\item Interesado
	\item Manejador
	\item Valoración
	\item Objetivo
	\item Principio
	\item Requerimiento
	\item Restricción
	\item Resultado
	\item Valor
	\item Significado
\end{itemize}